% Generated by Sphinx.
\def\sphinxdocclass{report}
\documentclass[letterpaper,10pt,oneside,openany]{sphinxmanual}
\usepackage[utf8]{inputenc}
\DeclareUnicodeCharacter{00A0}{\nobreakspace}
\usepackage{cmap}
\usepackage[T1]{fontenc}
\usepackage[english]{babel}
\usepackage{times}
\usepackage[Bjarne]{fncychap}
\usepackage{longtable}
\usepackage{sphinx}
\usepackage{multirow}

\addto\captionsenglish{\renewcommand{\figurename}{Fig. }}
\addto\captionsenglish{\renewcommand{\tablename}{Table }}
\floatname{literal-block}{Listing }



\title{NJU\_DMRG Documentation}
\date{November 21, 2015}
\release{1.0.0}
\author{Leo, Yang, Wang and Gu}
\newcommand{\sphinxlogo}{}
\renewcommand{\releasename}{Release}
\makeindex

\makeatletter
\def\PYG@reset{\let\PYG@it=\relax \let\PYG@bf=\relax%
    \let\PYG@ul=\relax \let\PYG@tc=\relax%
    \let\PYG@bc=\relax \let\PYG@ff=\relax}
\def\PYG@tok#1{\csname PYG@tok@#1\endcsname}
\def\PYG@toks#1+{\ifx\relax#1\empty\else%
    \PYG@tok{#1}\expandafter\PYG@toks\fi}
\def\PYG@do#1{\PYG@bc{\PYG@tc{\PYG@ul{%
    \PYG@it{\PYG@bf{\PYG@ff{#1}}}}}}}
\def\PYG#1#2{\PYG@reset\PYG@toks#1+\relax+\PYG@do{#2}}

\expandafter\def\csname PYG@tok@gd\endcsname{\def\PYG@tc##1{\textcolor[rgb]{0.63,0.00,0.00}{##1}}}
\expandafter\def\csname PYG@tok@gu\endcsname{\let\PYG@bf=\textbf\def\PYG@tc##1{\textcolor[rgb]{0.50,0.00,0.50}{##1}}}
\expandafter\def\csname PYG@tok@gt\endcsname{\def\PYG@tc##1{\textcolor[rgb]{0.00,0.27,0.87}{##1}}}
\expandafter\def\csname PYG@tok@gs\endcsname{\let\PYG@bf=\textbf}
\expandafter\def\csname PYG@tok@gr\endcsname{\def\PYG@tc##1{\textcolor[rgb]{1.00,0.00,0.00}{##1}}}
\expandafter\def\csname PYG@tok@cm\endcsname{\let\PYG@it=\textit\def\PYG@tc##1{\textcolor[rgb]{0.25,0.50,0.56}{##1}}}
\expandafter\def\csname PYG@tok@vg\endcsname{\def\PYG@tc##1{\textcolor[rgb]{0.73,0.38,0.84}{##1}}}
\expandafter\def\csname PYG@tok@m\endcsname{\def\PYG@tc##1{\textcolor[rgb]{0.13,0.50,0.31}{##1}}}
\expandafter\def\csname PYG@tok@mh\endcsname{\def\PYG@tc##1{\textcolor[rgb]{0.13,0.50,0.31}{##1}}}
\expandafter\def\csname PYG@tok@cs\endcsname{\def\PYG@tc##1{\textcolor[rgb]{0.25,0.50,0.56}{##1}}\def\PYG@bc##1{\setlength{\fboxsep}{0pt}\colorbox[rgb]{1.00,0.94,0.94}{\strut ##1}}}
\expandafter\def\csname PYG@tok@ge\endcsname{\let\PYG@it=\textit}
\expandafter\def\csname PYG@tok@vc\endcsname{\def\PYG@tc##1{\textcolor[rgb]{0.73,0.38,0.84}{##1}}}
\expandafter\def\csname PYG@tok@il\endcsname{\def\PYG@tc##1{\textcolor[rgb]{0.13,0.50,0.31}{##1}}}
\expandafter\def\csname PYG@tok@go\endcsname{\def\PYG@tc##1{\textcolor[rgb]{0.20,0.20,0.20}{##1}}}
\expandafter\def\csname PYG@tok@cp\endcsname{\def\PYG@tc##1{\textcolor[rgb]{0.00,0.44,0.13}{##1}}}
\expandafter\def\csname PYG@tok@gi\endcsname{\def\PYG@tc##1{\textcolor[rgb]{0.00,0.63,0.00}{##1}}}
\expandafter\def\csname PYG@tok@gh\endcsname{\let\PYG@bf=\textbf\def\PYG@tc##1{\textcolor[rgb]{0.00,0.00,0.50}{##1}}}
\expandafter\def\csname PYG@tok@ni\endcsname{\let\PYG@bf=\textbf\def\PYG@tc##1{\textcolor[rgb]{0.84,0.33,0.22}{##1}}}
\expandafter\def\csname PYG@tok@nl\endcsname{\let\PYG@bf=\textbf\def\PYG@tc##1{\textcolor[rgb]{0.00,0.13,0.44}{##1}}}
\expandafter\def\csname PYG@tok@nn\endcsname{\let\PYG@bf=\textbf\def\PYG@tc##1{\textcolor[rgb]{0.05,0.52,0.71}{##1}}}
\expandafter\def\csname PYG@tok@no\endcsname{\def\PYG@tc##1{\textcolor[rgb]{0.38,0.68,0.84}{##1}}}
\expandafter\def\csname PYG@tok@na\endcsname{\def\PYG@tc##1{\textcolor[rgb]{0.25,0.44,0.63}{##1}}}
\expandafter\def\csname PYG@tok@nb\endcsname{\def\PYG@tc##1{\textcolor[rgb]{0.00,0.44,0.13}{##1}}}
\expandafter\def\csname PYG@tok@nc\endcsname{\let\PYG@bf=\textbf\def\PYG@tc##1{\textcolor[rgb]{0.05,0.52,0.71}{##1}}}
\expandafter\def\csname PYG@tok@nd\endcsname{\let\PYG@bf=\textbf\def\PYG@tc##1{\textcolor[rgb]{0.33,0.33,0.33}{##1}}}
\expandafter\def\csname PYG@tok@ne\endcsname{\def\PYG@tc##1{\textcolor[rgb]{0.00,0.44,0.13}{##1}}}
\expandafter\def\csname PYG@tok@nf\endcsname{\def\PYG@tc##1{\textcolor[rgb]{0.02,0.16,0.49}{##1}}}
\expandafter\def\csname PYG@tok@si\endcsname{\let\PYG@it=\textit\def\PYG@tc##1{\textcolor[rgb]{0.44,0.63,0.82}{##1}}}
\expandafter\def\csname PYG@tok@s2\endcsname{\def\PYG@tc##1{\textcolor[rgb]{0.25,0.44,0.63}{##1}}}
\expandafter\def\csname PYG@tok@vi\endcsname{\def\PYG@tc##1{\textcolor[rgb]{0.73,0.38,0.84}{##1}}}
\expandafter\def\csname PYG@tok@nt\endcsname{\let\PYG@bf=\textbf\def\PYG@tc##1{\textcolor[rgb]{0.02,0.16,0.45}{##1}}}
\expandafter\def\csname PYG@tok@nv\endcsname{\def\PYG@tc##1{\textcolor[rgb]{0.73,0.38,0.84}{##1}}}
\expandafter\def\csname PYG@tok@s1\endcsname{\def\PYG@tc##1{\textcolor[rgb]{0.25,0.44,0.63}{##1}}}
\expandafter\def\csname PYG@tok@gp\endcsname{\let\PYG@bf=\textbf\def\PYG@tc##1{\textcolor[rgb]{0.78,0.36,0.04}{##1}}}
\expandafter\def\csname PYG@tok@sh\endcsname{\def\PYG@tc##1{\textcolor[rgb]{0.25,0.44,0.63}{##1}}}
\expandafter\def\csname PYG@tok@ow\endcsname{\let\PYG@bf=\textbf\def\PYG@tc##1{\textcolor[rgb]{0.00,0.44,0.13}{##1}}}
\expandafter\def\csname PYG@tok@sx\endcsname{\def\PYG@tc##1{\textcolor[rgb]{0.78,0.36,0.04}{##1}}}
\expandafter\def\csname PYG@tok@bp\endcsname{\def\PYG@tc##1{\textcolor[rgb]{0.00,0.44,0.13}{##1}}}
\expandafter\def\csname PYG@tok@c1\endcsname{\let\PYG@it=\textit\def\PYG@tc##1{\textcolor[rgb]{0.25,0.50,0.56}{##1}}}
\expandafter\def\csname PYG@tok@kc\endcsname{\let\PYG@bf=\textbf\def\PYG@tc##1{\textcolor[rgb]{0.00,0.44,0.13}{##1}}}
\expandafter\def\csname PYG@tok@c\endcsname{\let\PYG@it=\textit\def\PYG@tc##1{\textcolor[rgb]{0.25,0.50,0.56}{##1}}}
\expandafter\def\csname PYG@tok@mf\endcsname{\def\PYG@tc##1{\textcolor[rgb]{0.13,0.50,0.31}{##1}}}
\expandafter\def\csname PYG@tok@err\endcsname{\def\PYG@bc##1{\setlength{\fboxsep}{0pt}\fcolorbox[rgb]{1.00,0.00,0.00}{1,1,1}{\strut ##1}}}
\expandafter\def\csname PYG@tok@mb\endcsname{\def\PYG@tc##1{\textcolor[rgb]{0.13,0.50,0.31}{##1}}}
\expandafter\def\csname PYG@tok@ss\endcsname{\def\PYG@tc##1{\textcolor[rgb]{0.32,0.47,0.09}{##1}}}
\expandafter\def\csname PYG@tok@sr\endcsname{\def\PYG@tc##1{\textcolor[rgb]{0.14,0.33,0.53}{##1}}}
\expandafter\def\csname PYG@tok@mo\endcsname{\def\PYG@tc##1{\textcolor[rgb]{0.13,0.50,0.31}{##1}}}
\expandafter\def\csname PYG@tok@kd\endcsname{\let\PYG@bf=\textbf\def\PYG@tc##1{\textcolor[rgb]{0.00,0.44,0.13}{##1}}}
\expandafter\def\csname PYG@tok@mi\endcsname{\def\PYG@tc##1{\textcolor[rgb]{0.13,0.50,0.31}{##1}}}
\expandafter\def\csname PYG@tok@kn\endcsname{\let\PYG@bf=\textbf\def\PYG@tc##1{\textcolor[rgb]{0.00,0.44,0.13}{##1}}}
\expandafter\def\csname PYG@tok@o\endcsname{\def\PYG@tc##1{\textcolor[rgb]{0.40,0.40,0.40}{##1}}}
\expandafter\def\csname PYG@tok@kr\endcsname{\let\PYG@bf=\textbf\def\PYG@tc##1{\textcolor[rgb]{0.00,0.44,0.13}{##1}}}
\expandafter\def\csname PYG@tok@s\endcsname{\def\PYG@tc##1{\textcolor[rgb]{0.25,0.44,0.63}{##1}}}
\expandafter\def\csname PYG@tok@kp\endcsname{\def\PYG@tc##1{\textcolor[rgb]{0.00,0.44,0.13}{##1}}}
\expandafter\def\csname PYG@tok@w\endcsname{\def\PYG@tc##1{\textcolor[rgb]{0.73,0.73,0.73}{##1}}}
\expandafter\def\csname PYG@tok@kt\endcsname{\def\PYG@tc##1{\textcolor[rgb]{0.56,0.13,0.00}{##1}}}
\expandafter\def\csname PYG@tok@sc\endcsname{\def\PYG@tc##1{\textcolor[rgb]{0.25,0.44,0.63}{##1}}}
\expandafter\def\csname PYG@tok@sb\endcsname{\def\PYG@tc##1{\textcolor[rgb]{0.25,0.44,0.63}{##1}}}
\expandafter\def\csname PYG@tok@k\endcsname{\let\PYG@bf=\textbf\def\PYG@tc##1{\textcolor[rgb]{0.00,0.44,0.13}{##1}}}
\expandafter\def\csname PYG@tok@se\endcsname{\let\PYG@bf=\textbf\def\PYG@tc##1{\textcolor[rgb]{0.25,0.44,0.63}{##1}}}
\expandafter\def\csname PYG@tok@sd\endcsname{\let\PYG@it=\textit\def\PYG@tc##1{\textcolor[rgb]{0.25,0.44,0.63}{##1}}}

\def\PYGZbs{\char`\\}
\def\PYGZus{\char`\_}
\def\PYGZob{\char`\{}
\def\PYGZcb{\char`\}}
\def\PYGZca{\char`\^}
\def\PYGZam{\char`\&}
\def\PYGZlt{\char`\<}
\def\PYGZgt{\char`\>}
\def\PYGZsh{\char`\#}
\def\PYGZpc{\char`\%}
\def\PYGZdl{\char`\$}
\def\PYGZhy{\char`\-}
\def\PYGZsq{\char`\'}
\def\PYGZdq{\char`\"}
\def\PYGZti{\char`\~}
% for compatibility with earlier versions
\def\PYGZat{@}
\def\PYGZlb{[}
\def\PYGZrb{]}
\makeatother

\renewcommand\PYGZsq{\textquotesingle}

\begin{document}

\maketitle
\tableofcontents
\phantomsection\label{index::doc}



\chapter{How to Build Lattices and Momentum Spaces}
\label{index:welcome-to-nju-dmrg-s-documentation-lattice-and-hamiltonian}\label{index:how-to-build-lattices-and-momentum-spaces}

\section{Variaty Kinds of Lattices}
\label{index:variaty-kinds-of-lattices}\phantomsection\label{index:module-lattice.latticelib}\index{lattice.latticelib (module)}
To construct a specific type of lattice, you may use the function
\index{construct\_lattice() (in module lattice.latticelib)}

\begin{fulllineitems}
\phantomsection\label{index:lattice.latticelib.construct_lattice}\pysiglinewithargsret{\code{lattice.latticelib.}\bfcode{construct\_lattice}}{\emph{N}, \emph{lattice\_shape='`}, \emph{a=None}, \emph{catoms=None}, \emph{args=\{\}}}{}
Uniform construct method for lattice.
\begin{description}
\item[{N:}] \leavevmode
The size of lattice.

\item[{lattice\_shape:}] \leavevmode
The shape of lattice.
\begin{itemize}
\item {} 
`'            -\textgreater{} the anonymous lattice.

\item {} 
`square'      -\textgreater{} square lattice.

\item {} 
`honeycomb'   -\textgreater{} honeycomb lattice.

\item {} 
`triangular'  -\textgreater{} triangular lattice.

\item {} 
`chain'       -\textgreater{} a chain.

\end{itemize}

\item[{a:}] \leavevmode
The unit vector.

\item[{catoms:}] \leavevmode
The atoms in a unit cell.

\item[{args:}] \leavevmode
Other arguments,
\begin{itemize}
\item {} 
\emph{form} -\textgreater{} the form used in constructing honeycomb lattice.

\end{itemize}

\end{description}

\end{fulllineitems}


It will return a instance of one of the following derivative classes of \textless{}Lattice\textgreater{},
\index{Chain (class in lattice.latticelib)}

\begin{fulllineitems}
\phantomsection\label{index:lattice.latticelib.Chain}\pysiglinewithargsret{\strong{class }\code{lattice.latticelib.}\bfcode{Chain}}{\emph{N, a=1.0, catoms={[}0.0{]}}}{}
Lattice of Chain.

Chain(N,a=(1.),catoms={[}(0.,0.){]})
\index{kspace (lattice.latticelib.Chain attribute)}

\begin{fulllineitems}
\phantomsection\label{index:lattice.latticelib.Chain.kspace}\pysigline{\bfcode{kspace}}
The \textless{}KSpace\textgreater{} instance correspond to a chain.

\end{fulllineitems}


\end{fulllineitems}

\index{Square\_Lattice (class in lattice.latticelib)}

\begin{fulllineitems}
\phantomsection\label{index:lattice.latticelib.Square_Lattice}\pysiglinewithargsret{\strong{class }\code{lattice.latticelib.}\bfcode{Square\_Lattice}}{\emph{N, catoms={[}(0.0, 0.0){]}}}{}
Square Lattice, using C4v Group.

Square\_Lattice(N,catoms={[}(0.,0.){]})
\index{kspace (lattice.latticelib.Square\_Lattice attribute)}

\begin{fulllineitems}
\phantomsection\label{index:lattice.latticelib.Square_Lattice.kspace}\pysigline{\bfcode{kspace}}
Get the \textless{}KSpace\textgreater{} instance.

\end{fulllineitems}


\end{fulllineitems}

\index{Triangular\_Lattice (class in lattice.latticelib)}

\begin{fulllineitems}
\phantomsection\label{index:lattice.latticelib.Triangular_Lattice}\pysiglinewithargsret{\strong{class }\code{lattice.latticelib.}\bfcode{Triangular\_Lattice}}{\emph{N, catoms={[}(0.0, 0.0){]}}}{}
Triangular Lattice, using C6v Group.

Triangular\_Lattice(N,catoms={[}(0.,0.){]})
\index{kspace (lattice.latticelib.Triangular\_Lattice attribute)}

\begin{fulllineitems}
\phantomsection\label{index:lattice.latticelib.Triangular_Lattice.kspace}\pysigline{\bfcode{kspace}}
Get the \textless{}KSpace\textgreater{} instance.

\end{fulllineitems}


\end{fulllineitems}

\index{Honeycomb\_Lattice (class in lattice.latticelib)}

\begin{fulllineitems}
\phantomsection\label{index:lattice.latticelib.Honeycomb_Lattice}\pysiglinewithargsret{\strong{class }\code{lattice.latticelib.}\bfcode{Honeycomb\_Lattice}}{\emph{N}, \emph{form=1}}{}
HoneyComb Lattice class.

Honeycomb\_Lattice(N,form=1.)
\begin{description}
\item[{form:}] \leavevmode
The form of lattice.
\emph{1} -\textgreater{} traditional one with 0 point at a vertex, using C3v group.
\emph{2} -\textgreater{} the C6v form with 0 point at the center of hexagon, using C6v group.

\end{description}
\index{kspace (lattice.latticelib.Honeycomb\_Lattice attribute)}

\begin{fulllineitems}
\phantomsection\label{index:lattice.latticelib.Honeycomb_Lattice.kspace}\pysigline{\bfcode{kspace}}
Get the \textless{}KSpace\textgreater{} instance.

\end{fulllineitems}


\end{fulllineitems}


The base class \textless{}Lattice\textgreater{} is a special kind(derivative) of \textless{}Structure\textgreater{}, which is featured with repeated units.
To know more about the abilities of our \textless{}Lattice\textgreater{} and \textless{}Structure\textgreater{} instances,
\index{Structure (class in lattice.structure)}

\begin{fulllineitems}
\phantomsection\label{index:lattice.structure.Structure}\pysiglinewithargsret{\strong{class }\code{lattice.structure.}\bfcode{Structure}}{\emph{sites}}{}
Structure Base class. A collection of sites.

Structure(sites)
\begin{description}
\item[{sites:}] \leavevmode
Positions of sites.

\item[{groups:}] \leavevmode
The translation group and point groups imposed on this lattice.

\item[{\_\_kdt\_\_/\_\_kdt\_map\_\_:}] \leavevmode
The kd-tree for this structure, and the mapping for kdt index and site index.

\end{description}
\index{b1s (lattice.structure.Structure attribute)}

\begin{fulllineitems}
\phantomsection\label{index:lattice.structure.Structure.b1s}\pysigline{\bfcode{b1s}}
The nearest neighbor bonds.

\end{fulllineitems}

\index{b2s (lattice.structure.Structure attribute)}

\begin{fulllineitems}
\phantomsection\label{index:lattice.structure.Structure.b2s}\pysigline{\bfcode{b2s}}
The nearest neighbor bonds.

\end{fulllineitems}

\index{b3s (lattice.structure.Structure attribute)}

\begin{fulllineitems}
\phantomsection\label{index:lattice.structure.Structure.b3s}\pysigline{\bfcode{b3s}}
The nearest neighbor bonds.

\end{fulllineitems}

\index{findsite() (lattice.structure.Structure method)}

\begin{fulllineitems}
\phantomsection\label{index:lattice.structure.Structure.findsite}\pysiglinewithargsret{\bfcode{findsite}}{\emph{pos}, \emph{tol=1e-05}}{}
Find the site at specific position.
\begin{description}
\item[{pos:}] \leavevmode
The position of the site.

\item[{tol:}] \leavevmode
The position tolerence.

\item[{\emph{return}:}] \leavevmode
The site index.

\end{description}

\end{fulllineitems}

\index{getbonds() (lattice.structure.Structure method)}

\begin{fulllineitems}
\phantomsection\label{index:lattice.structure.Structure.getbonds}\pysiglinewithargsret{\bfcode{getbonds}}{\emph{n}}{}
Get n-th nearest bonds.
\begin{description}
\item[{n: }] \leavevmode
Specify which set of bonds.

\item[{\emph{return}:}] \leavevmode
A \textless{}BondCollection\textgreater{} instance.

\end{description}

\end{fulllineitems}

\index{initbonds() (lattice.structure.Structure method)}

\begin{fulllineitems}
\phantomsection\label{index:lattice.structure.Structure.initbonds}\pysiglinewithargsret{\bfcode{initbonds}}{\emph{nmax=3}, \emph{K=None}, \emph{tol=1e-05}, \emph{leafsize=30}}{}
Initialize the distance(bond) mesh, and classify it by onsite,1st,2ed,3rd ... nearest neighbours.
\begin{description}
\item[{nmax:}] \leavevmode
Up to nmax-th neighbor will be considered.

Note: if nmax\textless{}0, it will initialize the tree for query but will not initialize bonds.

\item[{K: }] \leavevmode
Number of neighbors calculated through cKD Tree, should be \textgreater{}= number of sites up to nmax-th neighbor.

\item[{tol:}] \leavevmode
The bond vector tolerence.

\item[{leafsize:}] \leavevmode
The leafsize of kdtree.

\item[{\emph{return}:}] \leavevmode
A list of bond vectors, the elements are 0,1,2...-th neightbors.

\end{description}

\end{fulllineitems}

\index{load\_bonds() (lattice.structure.Structure method)}

\begin{fulllineitems}
\phantomsection\label{index:lattice.structure.Structure.load_bonds}\pysiglinewithargsret{\bfcode{load\_bonds}}{\emph{filename=None}}{}
Load bonds.
\begin{description}
\item[{filename:}] \leavevmode
The target filename.

\end{description}

\end{fulllineitems}

\index{measure() (lattice.structure.Structure method)}

\begin{fulllineitems}
\phantomsection\label{index:lattice.structure.Structure.measure}\pysiglinewithargsret{\bfcode{measure}}{\emph{i}, \emph{j}, \emph{k=2}}{}
Measure the `true' distance between sites at ri and rj. 
Here, the main problem is the periodic bondary condition.
\begin{description}
\item[{i/j: }] \leavevmode
index of start/end atom.

\item[{k:}] \leavevmode
The maximum times of translation group imposed on r2.

\item[{\emph{return}:}] \leavevmode
({\color{red}\bfseries{}\textbar{}r\textbar{}},r), absolute distance and vector distance.

\end{description}

\end{fulllineitems}

\index{nsite (lattice.structure.Structure attribute)}

\begin{fulllineitems}
\phantomsection\label{index:lattice.structure.Structure.nsite}\pysigline{\bfcode{nsite}}
Number of sites

\end{fulllineitems}

\index{query() (lattice.structure.Structure method)}

\begin{fulllineitems}
\phantomsection\label{index:lattice.structure.Structure.query}\pysiglinewithargsret{\bfcode{query}}{\emph{pos}, \emph{k}, \emph{**kwargs}}{}
Query the K-nearest sites near specific site.
\begin{description}
\item[{pos:}] \leavevmode
The position of target site.

\item[{k:}] \leavevmode
The number of neighbors.

\item[{{\color{red}\bfseries{}**}kwargs:}] \leavevmode
Key word arguments for cKDTree.

\item[{\emph{return}:}] \leavevmode
int32 array of length k, the indices of neighbor sites.

\end{description}

\end{fulllineitems}

\index{save\_bonds() (lattice.structure.Structure method)}

\begin{fulllineitems}
\phantomsection\label{index:lattice.structure.Structure.save_bonds}\pysiglinewithargsret{\bfcode{save\_bonds}}{\emph{filename=None}, \emph{nmax=3}}{}
Save bonds.
\begin{description}
\item[{filename:}] \leavevmode
The target filename.

\item[{nmax:}] \leavevmode
Up to nmax-th neighnors are saved.

\end{description}

\end{fulllineitems}

\index{show\_bonds() (lattice.structure.Structure method)}

\begin{fulllineitems}
\phantomsection\label{index:lattice.structure.Structure.show_bonds}\pysiglinewithargsret{\bfcode{show\_bonds}}{\emph{nth=(1}, \emph{)}, \emph{plane=(0}, \emph{1)}, \emph{color='r'}}{}
Plot the structure.
\begin{description}
\item[{plane:}] \leavevmode
project to the specific plane if it is a 3D structre.
Default is (0,1) - `x-y' plane.

\item[{nth:}] \leavevmode
the n-th nearest bonds are plotted. It should be a tuple.
Default is (1,) - the nearest neightbor.

\item[{c:}] \leavevmode
color, default is `r' -red.

\end{description}

\end{fulllineitems}

\index{show\_sites() (lattice.structure.Structure method)}

\begin{fulllineitems}
\phantomsection\label{index:lattice.structure.Structure.show_sites}\pysiglinewithargsret{\bfcode{show\_sites}}{\emph{plane=(0}, \emph{1)}, \emph{color='r'}}{}
Show the sites in this structure.
\begin{description}
\item[{color:}] \leavevmode
The color, string.

\end{description}

\end{fulllineitems}

\index{usegroup() (lattice.structure.Structure method)}

\begin{fulllineitems}
\phantomsection\label{index:lattice.structure.Structure.usegroup}\pysiglinewithargsret{\bfcode{usegroup}}{\emph{g}}{}
Apply a group on this lattice.
\begin{description}
\item[{g: }] \leavevmode
A \textless{}Group\textgreater{} instance.

\end{description}

\end{fulllineitems}

\index{vdim (lattice.structure.Structure attribute)}

\begin{fulllineitems}
\phantomsection\label{index:lattice.structure.Structure.vdim}\pysigline{\bfcode{vdim}}
Dimention of vector space.

\end{fulllineitems}


\end{fulllineitems}

\index{Lattice (class in lattice.lattice)}

\begin{fulllineitems}
\phantomsection\label{index:lattice.lattice.Lattice}\pysiglinewithargsret{\strong{class }\code{lattice.lattice.}\bfcode{Lattice}}{\emph{name, a, N, catoms={[}(0.0, 0.0){]}}}{}
Lattice Structure, which contains tranlation of cells.

Lattice(name,a,N,catoms={[}(0.,0.){]})
\begin{description}
\item[{name:}] \leavevmode
The name of this latice.

\item[{a:}] \leavevmode
Lattice vector

\item[{N: }] \leavevmode
Number of cells

\item[{catoms:}] \leavevmode
Atoms in one cell.

\item[{lmesh:}] \leavevmode
The sites reshaped according to the lattice config (Nx,Ny, ..., ncatom).

\end{description}
\index{cbonds (lattice.lattice.Lattice attribute)}

\begin{fulllineitems}
\phantomsection\label{index:lattice.lattice.Lattice.cbonds}\pysigline{\bfcode{cbonds}}
Get a list of bonds within a unit cell.

\end{fulllineitems}

\index{dimension (lattice.lattice.Lattice attribute)}

\begin{fulllineitems}
\phantomsection\label{index:lattice.lattice.Lattice.dimension}\pysigline{\bfcode{dimension}}
The dimension of lattice.

\end{fulllineitems}

\index{findsite() (lattice.lattice.Lattice method)}

\begin{fulllineitems}
\phantomsection\label{index:lattice.lattice.Lattice.findsite}\pysiglinewithargsret{\bfcode{findsite}}{\emph{pos}, \emph{tol=1e-05}}{}
Get the lattice indices from the position.
\begin{description}
\item[{pos: }] \leavevmode
the position r.

\item[{tol:}] \leavevmode
the tolerence of atom position.

\item[{\emph{return}:}] \leavevmode
an array of lattice index.

\end{description}

\end{fulllineitems}

\index{index2l() (lattice.lattice.Lattice method)}

\begin{fulllineitems}
\phantomsection\label{index:lattice.lattice.Lattice.index2l}\pysiglinewithargsret{\bfcode{index2l}}{\emph{index}}{}
Get lattice indices (n1,n2,...,atom index in cell) from site index.
\begin{description}
\item[{index:}] \leavevmode
the site index.

\end{description}

\end{fulllineitems}

\index{kspace (lattice.lattice.Lattice attribute)}

\begin{fulllineitems}
\phantomsection\label{index:lattice.lattice.Lattice.kspace}\pysigline{\bfcode{kspace}}
Get the KSpace instance correspond to this lattice.

\end{fulllineitems}

\index{l2index() (lattice.lattice.Lattice method)}

\begin{fulllineitems}
\phantomsection\label{index:lattice.lattice.Lattice.l2index}\pysiglinewithargsret{\bfcode{l2index}}{\emph{lindex}}{}
Get the site index from lattice indices.
lindex:
\begin{quote}

lattice index - (n1,n2,...,atom index in cell)
\end{quote}

\end{fulllineitems}

\index{ncatom (lattice.lattice.Lattice attribute)}

\begin{fulllineitems}
\phantomsection\label{index:lattice.lattice.Lattice.ncatom}\pysigline{\bfcode{ncatom}}
number of atoms within a unit cell.

\end{fulllineitems}

\index{showcell() (lattice.lattice.Lattice method)}

\begin{fulllineitems}
\phantomsection\label{index:lattice.lattice.Lattice.showcell}\pysiglinewithargsret{\bfcode{showcell}}{\emph{bondindex=(1}, \emph{2)}, \emph{plane=(0}, \emph{1)}, \emph{color='r'}, \emph{offset=None}}{}
Plot the cell structure.
\begin{description}
\item[{bondindex:}] \leavevmode
the bondindex-th nearest bonds are plotted. It should be a tuple.
Default is (1,2) - the nearest, and second nearest neightbors.

\item[{plane:}] \leavevmode
project to the specific plane if it is a 3D structre.
Default is (0,1) - `x-y' plane.

\item[{color:}] \leavevmode
color, default is `r' -red.

\item[{offset:}] \leavevmode
The offset of the sample cell.

\end{description}

\end{fulllineitems}

\index{siteconfig (lattice.lattice.Lattice attribute)}

\begin{fulllineitems}
\phantomsection\label{index:lattice.lattice.Lattice.siteconfig}\pysigline{\bfcode{siteconfig}}
The site configuration, taking catoms into account.

\end{fulllineitems}


\end{fulllineitems}



\chapter{How to Write Down Hamiltonians}
\label{index:how-to-write-down-hamiltonians}

\renewcommand{\indexname}{Python Module Index}
\begin{theindex}
\def\bigletter#1{{\Large\sffamily#1}\nopagebreak\vspace{1mm}}
\bigletter{l}
\item {\texttt{lattice.latticelib}}, \pageref{index:module-lattice.latticelib}
\end{theindex}

\renewcommand{\indexname}{Index}
\printindex
\end{document}
